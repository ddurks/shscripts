\documentclass[letterpaper]{article}

\title{Reading08}
\date{March 14, 2016}
\author{David Durkin}

\usepackage{graphicx}
\usepackage{hyperref}
\usepackage[margin=1in]{geometry}

\begin{document}

\maketitle

This document provides an overview of how I completed Reading08


\section*{Overview}

In this reading I created an dice rolling script and another script 
which rolled dice 1000 times and outputted the results. Then, I used 
gnuplot in order to create a plot of said data.


\section*{Rolling Dice}

The rolling dice script is implemented as follows:

\begin{verbatim}
	#!/bin/sh

	while getopts r:s: name
	do
		case $name in
				r)ropt=$OPTARG;;
				s)sopt=$OPTARG;;
				*)echo "Invalid arg";;
		esac
	done

	if [ $ropt ]; then
		if [ $sopt ]; then
			for i in `seq $ropt`; do
				seq 1 $sopt | shuf | head -1
			done
		else
			for i in `seq $ropt`; do
				seq 1 6 | shuf | head -1
			done
		fi
	elif [ $sopt ]; then
		if [ $ropt ]; then
			for i in `seq $ropt`; do
				seq 1 $sopt | shuf | head -1
			done
		else
			for i in {1..10}; do
				seq 1 $sopt | shuf | head -1
			done
		fi
	else
		for i in {1..10}; do
			seq 1 6 | shuf | head -1
		done
	fi
\end{verbatim}

Here, I am using getopts to obtain the arguments, and then using a series of seq, shuf, and head statements to get the proper outputs dependent on the proper variables


\section*{Experiment}

The experiment script consists of the following code:

\begin{verbatim}
	#!/bin/sh

	./roll_dice.sh -r 1000 | awk 'NR == FNR { ++count[$1] } END{for(i in count){print i,'\t', count[i]};}' | sort > results.dat
\end{verbatim}

This simply implements the roll dice program and filters the output using awk


\section*{Results}

The resulting data:

\input{results.dat}


The resulting plot using {\bf gnuplot}:

\begin{figure}[h!]
\centering
\includegraphics[width=5in]{results.png}
\caption{Plot}
\label{fig:Plot}
\end{figure}

\end{document}
