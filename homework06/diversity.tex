\documentclass[letterpaper]{article}

\title{Diversity in CS}
\date{March 18, 2016}
\author{David Durkin}

\usepackage{graphicx}
\usepackage{hyperref}
\usepackage[margin=1in]{geometry}

\begin{document}

\maketitle

This document provides an analysis of the Diversity in Computer Science at Notre Dame.


\section*{Overview}

	In order to first sift out the gender and ethnicity information into a
readable format, I first created a program called filter_demographics. This shell script used primarily awk in order to create a file with a readable format outlining the gender information over the years. Then I created a gnuplot readable file called results.dat. From there I created two graphs, one for diversity and one for gender. The main thing that I took away from analyzing the data is that diversity is an area that needs improvement in CS at Notre Dame.


\section*{Methodology}

The filter_demographics script is implemented as follows:

\begin{verbatim}
	#!/bin/sh

	awk 'BEGIN{ FS=","; } 
	NR == FNR 
	{ ++gender13[$1]; ++race13[$2];
	++gender14[$3]; ++race14[$4]; 
	++gender15[$5]; ++race15[$6]; 
	++gender16[$7]; ++race16[$8]; 
	++gender17[$9]; ++race17[$10]; 
	++gender18[$11]; ++race18[$12]; }  
	END{
		print "year thirteen";
		for(i in gender13){
			print i,'\t', gender13[i]
		};
		race13[U]=0;
		for(i in race13){
			print i,'\t', race13[i]
		};
		print "year fourteen";
		for(i in gender14){
			print i,'\t', gender14[i]
		};
		for(i in race14){
			print i,'\t', race14[i]
		};
		print "year fifteen";
		for(i in gender15){
			print i,'\t', gender15[i]
		};
		for(i in race15){
			print i,'\t', race15[i]
		};
		print "year sixteen";
		for(i in gender16){
			print i,'\t', gender16[i]
		};
		for(i in race16){
			print i,'\t', race16[i]
		};
		print "year seventeen";
		for(i in gender17){
			print i,'\t', gender17[i]
		};
		for(i in race17){
			print i,'\t', race17[i]
		};
		print "year eighteen";
		for(i in gender18){
			print i,'\t', gender18[i]
		};
		for(i in race18){
			print i,'\t', race18[i]
		};
	}' | sed '/^ /d' | sed '/201.*/d' | tail -54
\end{verbatim}

Here, I am using associative arrays to filter out and sort the proper information, and then a series of for loops and sed statements to print out the desired output.


\section*{Analysis}

My graphs and tables

Gender:

\begin{gender}
\begin{tabular}{ |c|c|c| }
	\hline
	Year & Female & Male \\
	\hline
	2013 & 14 & 49 \\
	\hline
	2014 & 12 & 44 \\
	\hline
	2015 & 16 & 58 \\
	\hline
	2016 & 19 & 60 \\
	\hline
	2017 & 26 & 65 \\
	\hline
	2018 & 36 & 90 \\
	\hline
\end{tabular}
\end{gender}

\begin{figure}[h!]
\centering
\includegraphics[width=5in]{gender.png}
\caption{Plot}
\label{fig:Plot}
\end{figure}

From this graph it is obvious that Computer science is dominated by males, although in recent years more and more women are becoming involved.

Ethnicity:

\begin{ethnicity}
\begin{tabular}{ |c|c|c|c|c|c|c| }
	\hline
	Year & African-American & Caucasian & Native-American & Oriental & Hispanic & Multiple-Selection & Undeclared \\
	\hline
	2013 & 3 & 43 & 1 & 7 & 7 & 2 & 0\\
	\hline
	2014 & 2 & 43 & 1 & 5 & 4 & 1 & 0 \\
	\hline
	2015 & 4 & 47 & 1 & 9 & 10 & 1 & 2 \\
	\hline
	2016 & 1 & 53 & 7 & 9 & 9 & 0 & 0 \\
	\hline
	2017 & 5 & 60 & 5 & 12 & 3 & 6 & 0 \\
	\hline
	2018 & 3 & 91 & 4 & 8 & 12 & 8 & 0 
	\hline
\end{tabular}
\end{ethnicity}

\begin{figure}[h!]
\centering
\includegraphics[width=5in]{ethnicity.png}
\caption{Plot}
\label{fig:Plot}
\end{figure}

From this data it is clear that Notre Dame CS is dominated by Caucasian males. This, in my opinion, needs to change.

\section*{Discussion}

I believe that the Computer Science and Engineering does a great job creating a welcoming environment for all people, and I feel that the heart of the problem with Diversity must lie either in the University itself or in some other variable that has to do with the content or culture of Computer Science overall.

\end{document}
